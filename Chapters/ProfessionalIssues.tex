\chapter{Professional and Ethical Issues}
\par During the development and evaluation of the system, great care has been taken to follow the Code of Conduct\footnote{\url{http://www.bcs.org/category/6030}} and Code of Good Practice\footnote{\url{http://www.bcs.org/upload/pdf/cop.pdf}} issued by the British Computer Society. This must be done to avoid serious legal and ethical problems.

\par Importance has been given to state explicitly which software libraries and academic papers have been used throughout the development. The information and services of these have been used and modified to meet the requirements of the project. All the software produced consists of my own work, except where it is explicitly said otherwise. 

\par As the system is intended to be released as open-source, it has been confirmed that the libraries allow this. If the system were to be used commercially, a license would have to be produced that would encapsulate the project and its libraries.

\par The developed system does not collect user data, thereby abiding to the public interest of the Code of Conduct. In addition, for the evaluation phase it was checked whether ethical approval would be required. It is not. Since the test participants are kept completely anonymous, and the data sources used, are publicly available.

\par It should be noted that if the system is moved onto a server to produce timelines from large sets of documents, the processing power used would increase substantially, and may affect other systems running on the same machine. This can be avoided by modifying the settings of the system, e.g. the number of threads used, to reduce the number of processes running.

