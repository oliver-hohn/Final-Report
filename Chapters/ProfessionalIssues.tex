\chapter{Professional and Ethical Issues}
Either in a seperate section or throughout the report demonstrate that you are aware of the \textbf{Code of Conduct \& Code of Good Practice} issued by the British Computer Society and have applied their principles, where appropriate, as you carried out your project.
-no user data is collected (public interest)
-project will be open source so the licenses applied can be used, however commerical would require a specific license
-how dealt with ethical approval? (newspapers used, etc.)
-in analysis kept testers anonymous
-if used on cloud and use innapropriate setting smay use a lot of resources
\par During the development and evaluation of the system, great care has been taken to follow the Code of Conduct\footnote{\url{http://www.bcs.org/category/6030}} and Code of Good Practice\footnote{\url{http://www.bcs.org/upload/pdf/cop.pdf}} issued by the British Computer Society. This must be done to avoid serious legal and ethical problems.
\par Importance has been given to state explicitly which software libraries and academic papers have been used throughout the development. Both have been changed to match the requirements of the project. The software produced consists of my own work, except where it is explicitly said otherwise in the documentation. 
\par As the system is intended to be released as open-source for further development, it has been confirmed that the third-party libraries allow this. If the system were to be used commercially, a license would have to be produced that would encapsulate the project and its libraries.
\par The developed system does not collect user data, thereby abiding to the public interest of the Code of Conduct. In addition, for the evaluation phase it was checked whether an ethical approval would be required. In this case, it is not required, as the test users are kept completely anonymous, and the data sources are publicly available.
\par It should be noted that if the system is moved onto a server to produce timelines from large sets of documents, the processing power used would increase substantially, and may affect other systems running on the same machine. This can be avoided by modifying the settings of the system, e.g. the number of threads used, to reduce the number of processes running.

