\chapter{Requirements \& Specification}
\section{Brief}
\par The system should take as input a set of documents, process them autonomously (i.e. without the user's involvement), and produce a graphical representation of events in the documents. This requires the identification of sentences in the text that contain dates, providing an exact date (to compare to other events), identifying subjects and providing a summary of the sentence. Requirements are generated from this.

\section {Requirements}
\subsection{Functional Requirements}
\par The functional requirements of a system are behaviours a system should have\footnote{\url{http://reqtest.com/requirements-blog/functional-vs-non-functional-requirements/}}. In this project, the functional requirements are as follows:
\begin{enumerate}
\item Process documents of different file types (e.g. .pdf, .txt, and .docx).
\item Identify dates and subjects in text.
\item Summarize sentences.
\item Produce a graphical timeline of the events in the input documents.
\item Modify/Delete events in the timeline.
\item Travel between timeline and relevant document.
\item Save timeline (as .pdf or .JSON).
\end{enumerate} 
\par As the software will require as input documents, the 3 most used document file types\footnote{\url{http://www.computerhope.com/issues/ch001789.htm}} should be allowable file types in the system. Since the aim is to identify events in the input text, and events are identified temporal expressions, then it is necessary to identify these. The subjects of a sentence are required to  aid the description of an event. Subjects, in this case, include names of people, locations, and quantities of money (i.e. key words).
\par The resulting system should enable users to modify events as it can be the case that the summary, subjects, or date determined by the system are wrong. Providing the ability to edit the events would allow the user to correct these mistakes. 
\par The final two requirements do not affect the processing of the system, but are advantageous to users. Being able to switch between timeline and document will provide the ability to go from a general description of an event to the actual, full-detail, and in context description (which is the original sentence of the event). Providing a save to PDF ability allows the timelines to be included in documents, as the PDF files can be merged. More interestingly, producing an intermediate JSON output makes the system compatible with 3rd-party applications that can provide other graphical representations and/or process further the data. Providing two graphical representations of a timeline (in the system and as a PDF) along with a JSON representation should allow the system to be integrated in documents, reports and other applications.

\subsection{Non-Functional Requirements}
\par Non-functional requirements of a project are descriptions of how the system must perform the functional requirements, and the qualities the system should have. In this project, the non-functional requirements are as follows:
\begin{enumerate}
\item A responsive and intuitive UI (Visibility).
\item Reasonable output time (Efficient).
\item Identify the majority of events (Effective).
\end{enumerate}
\par These three requirements will be evaluated to determine if these are met in the produced system. 
\par Since the system should be used by any kind of user, with no required technical knowledge, it should be intuitive for the user to know how to use it, i.e. the system should be usable. The user if the system will be discussed later. It would be unreasonable that the resulting system is extremely slow. Such would be the case if on an input $n$ it would require a much larger (exponential) time to complete. 
\par Efficiency relates to the task of identifying events. Identifying events is the most important non-functional requirement, and one of the most important general requirements of the system. The system should be able to extract simple events in text where the full date is mentioned, but also be able to extract more complicated events where the temporal expressions are ambiguous. 
\par Extraction and inference of dates on temporal expressions will be further discussed later. This issue is relevant as in some cases it is known that an event occurred after another, but the specific date cannot be determined. Thereby, producing a timeline of linked events, where an event appears after another not because its exact date suggests so, but the context would allow for the event to be linked in such way. This requires changing models in  established NLP tools, and thus will be discussed in the Future Works chapter.

\section{Limitations}
\par The greatest limitation of the project is time, both in its development and in the execution of the system. As the project time is limited, compromises must be made. For example, a noisy-channel neural-network summary system would produce different plausible summaries for a given text, of which one should be a reasonable summary. However, as mentioned previously, noisy-channel models require a large amount of annotated data (and thus are domain-dependent). Thereby, providing this set of data would require more development time, which would not allow for the completion of the project. In addition, loading large models of data to summarize one sentence, would have a substantial impact in the running time of the system. From the users point of view, if the system is not significantly faster than producing timelines manually, then they would prefer to do them manually as they are more accurate.

\par A limitation in all NLP projects is the technology. Modelling context in systems is extremely difficult and non-trivial \cite{iwanskazadrozny1997}. For this project, it is a clear issue, that for some inputs, poor timelines will be produced as the context of the text was not fully modelled, and thus understood by the system. The issue can be applied to ambiguous temporal expressions, where the exact date referenced in the sentence cannot be determined. Solutions to these problems proposed by NLP researchers include looking at sentences related to each other (instead of independently) and thereby linking references. However, this still does not mimic the context understood by humans during read or spoken interactions.

\section{Additional Aims} 
\par An aim of the project is to make it open-source by providing it on GitHub along with a license. The license would allow anyone to use the system. This is beneficial as the system can be integrated in other programs (directly, as a library, or indirectly, through JSON). All the libraries used allow for the system to be open-source.

