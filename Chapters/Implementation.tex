\chapter{Implementation}
-How implemented
\section{Approach}
\par The development approach taken focused on the business-logic, or back-end, of the system. From the two main possible development methodologies, Waterfall and Agile, the latter was used. In Waterfall, the devleopment process is a sequential process, where the dvelopment is considered as a sequence of phases that are completed one after the other (cite). In Agile, the focus in adaptive planning, evolutionary development, and continous improvement. The advantage of using an Agile approach over a Waterfall approach is that new features can be mplemented into the system easier. 
\par The most used way of using Agile methodology is through Sprint cycles. These are short development cycles, where a set of features must be implemented, alongside their tests to ensure the correctness of features(discussed in the Testing section). For this project, the cycles combined with the supervisor meetings. Since in them, the new features that were implemented were discussed alongside the new features that were to be implemented in the next cycle. 
\par A clear example of the advantage of this system was when the a new timeline view was suggested. In this new view, rather than having the events row by row, events that occurred during the same time period should be grouped. In addition, events that happened within that time period should be encapsulated by the larger events. This could be implemented in the system, due to the separation of the business logic and the view, and the development approach used. In a Waterfall model, the development is more structured, and thereby it is extremely useful for static requirements, i.e. requirements that will not change. However, in this case it would have caused issues in implementing the new view as it would require going up the Waterfall if the view of the system had already been implemented, or waiting until that step of the waterfall had been reached.
\par While the Agile methodology is mostly used in software development teams, it can be applied to single development projects. Since the structure allows for reviews of features which can be matched with supervisor meetings, and changes in the requirements of the project.
\section{Tools & Software Libraries}
-development environment
-why used that environment
-software libraries (include an example use)
-why
\section{Issues}
-ner dates (explain, then present how to solve it through examples of code)
-new timeline view (present how to solve it through examples of code)
\section{Testing}
\section{UI}
