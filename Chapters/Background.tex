\chapter{Background}
The background should set the project into context by motivating the subject matter and relating it to existing published work. The background will include a critical evaluation of the existing literature in the area in which your project work is based and should lead the reader to understand how your work is motivated by and related to existing work.
-the problem with processing text (Twitter experiment)

\section{NLP}
-explain what NLP is
-what parts of NLP are involved
-cite
\section{Data Processing and Representation}
-explain the event
-what are the options for the summary (Neural Networks vs Decision-Based)
-give an algorithm for determining the summary, with an example
-explain the date problem, with example (determine that it uses an ISO standard)



//defintion of an event
An event is given by its date(s), subjects and a short summary of the sentence that produced it. An event can have more than 1 date if it is considered to happen in a range of dates. For example, an event that happened in the 1980s would have two dates, one for the start date: 1980-01-01, and one for the end date: 1989-12-31. While an event that happened just on one day would have just one date. The subjects of an event are given by the "person, place, thing, or idea that is doing or being something" (Grammar.ccc).
