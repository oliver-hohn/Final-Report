\chapter{Conclusion and Future Work}

The project's conclusions should list the key things that have been learnt as a consequence of engaging in your project work. For example, ``The use of overloading in C++ provides a very elegant mechanism for transparent parallelisation of sequential programs'', or ``The overheads of linear-time n-body algorithms makes them computationally less efficient than $O(n \log n)$ algorithms for systems with less than 100000 particles''. Avoid tedious personal reflections like ``I learned a lot about C++ programming...'', or ``Simulating colliding galaxies can be real fun...''. It is common to finish the report by listing ways in which the project can be taken further. This might, for example, be a plan for turning a piece of software or hardware into a marketable product, or a set of ideas for possibly turning your project into an MPhil or PhD.

-what have you learned?
-how can the project be carried further (neural net for summary, building on the StanfordCoreNLP for detas depending on others)
\section{Conclusion of Project}
\section{Future Work}
-neural bet, cloud, building on tool to link to ambigious dates that are related to other possible known ones, edit events (machine-learning in addition to neural net)
\par Future works consist of areas of possible development that would improve the overall effectiveness of the system. The areas are described below. Implementing any of these would be an improvement, however the challenge is in being able to combine them all to produce a system that can be used at a commercial level, and would be unrivalled by other systems.
\par \textbf{Neural Net} - As mentioned in the Background Chapter, there is a heavy use of Neural Nets for text summarization, as can be seen from the works of \cite{chopraaulirush2016} and \cite{rushchopraweston2015}. These are often combined with noisy-channel models that use data sets for statistcal computation of summaries. The main benefit of such a system would be to provide better summaries. The reason for them not being used in the project is the large data set required and the amount of computation done to produce a summary of one sentence.
\par \textbf{Machine Learning} - Machine learning allows a system to become more accurate by incrementing its data set, and it then being used to produce outputs (cite). It is similar to Neural Nets. In this system it could be used in the production of events,in the following. When a user edits an event it can be assummed that they produced a corrected event. This data can then be used and considered in the creation of other events, to produce more precise events. Thereby the system would improve over time, as it would produce more precise and reliable timelines.
\par \textbf{Cloud} - A cloud allows a service to be provided independent of its software and hardware (cite). This is done by deploying the system on a remote server, and give clients an interface to interact with it. Thereby it is not required for the client to have a powerful machine, and it does not affect their systems performance. The downfall is the cost. However, if this would be created the performance could be improved as hardware could be tuned to be efficient for this purpose and it could be linked with Machine-Learning and Neural Nets to allow the system to impove in its event identification and production.
\par \textbf{Extending StanfordCoreNLP} - An issue in the tool used, is that it does not attempt to link ambigious temporal expressions. For example, it may be known that an event occurred before another. For example, a person has to be born first before they can work \cite{mccloskymanning2012}. In the produced system, if it cannot determine an exact data then an ambigious date will be give, for example for both born and working it could produce a "PAST\_REF". This would then cause the system to assign both events the same range of dates, i.e. 0001-01-01 to the reference point used for the document. However, these can be made more precisely as a person has to be born first and be over the age of 16 to work. Thereby the range of dates for the work event would be smaller than that of being born. This problem arises, as the tool and the produced system consider each sentence independently of each other. It would be benefitial if the events could be linked one after the other, such that it may not be known when someone was born or when they worked, but that the event of them working is after the event of them being born. This would require building on the NLP tool used in the system, and would provide for a more effective system.